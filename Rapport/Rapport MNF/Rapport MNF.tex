\documentclass[11pt,a4paper]{report}

\usepackage[utf8]{inputenc}
\usepackage[french]{babel}
\usepackage[T1]{fontenc}
\usepackage{amsmath}
\usepackage{amsfonts}
\usepackage{amssymb}

\title{Rapport Méthode Numérique en Finance}

\begin{document}

\chapter{Modèle de Black \& Scholes et Equations aux dérivées partielles}

\section{Introduction}

	L'objectif de cette partie est de pricer une option européenne selon le modèle de Black \& Scholes. Pour cela nous utiliserons tout d'abord la formule exacte du prix du call puis nous comparerons ce résultat avec la résolution de l'équation différentielle sous différents schémas temporels.\\
	Nous utiliserons les paramètres suivants dans la suite de ce chapitre :

\begin{align*}
  	Si &= 150& &\text{ la valeur actuelle de l'action sous-jacente} \\
  	K &= 100& &\text{ le prix du Strike} \\
	r &= 0.05& &\text{ le taux d'intérêt sans risque} \\
	sig &= 0.2&  &\text{ la volatilité du prix de l'action} \\
	T &= 1& &\text{ le temps à la maturité} \\
	L &= log(1000)& &\text{ la valeur maximale du prix de l'option}
\end{align*}




\end{document}